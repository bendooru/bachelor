\documentclass[
  paper=a4,
  DIV=14,
  fontsize=12pt,
  titlepage,
  bibliography=totoc,
  pagesize=pdftex
]{scrartcl}
\usepackage[utf8]{inputenc}
\usepackage[T1]{fontenc}
\usepackage[english]{babel}
\usepackage[autostyle=true]{csquotes}
\usepackage{enumitem}
\usepackage{mathtools,amsfonts,amssymb,amsthm,mathrsfs}
\usepackage{lmodern}
\usepackage{dsfont}

\usepackage[
  activate={true,nocompatibility},
  kerning=true,
  spacing=true
]{microtype}

\usepackage[pdftex,hidelinks]{hyperref}

\usepackage[backend=biber,style=alphabetic]{biblatex}
\addbibresource{bachelor.bib}

% serif > sans-serif
\setkomafont{sectioning}{\rmfamily\bfseries\boldmath}
\setkomafont{descriptionlabel}{\rmfamily\bfseries\boldmath}
% prefix all numbers with section
\numberwithin{figure}{section}
\numberwithin{equation}{section}
\numberwithin{table}{section}

% set default numbered list style to (i), (ii), ...
\setlist[enumerate,1]{label=(\roman*)}

% use mathds instead of mathbb
\newcommand*\setZ{\mathds{Z}}
\newcommand*\setR{\mathds{R}}
\newcommand*\setC{\mathds{C}}
\newcommand*\setQ{\mathds{Q}}
\newcommand*\setN{\mathds{N}}
\newcommand*\setK{\mathds{K}}
\newcommand*\setP{\mathds{P}}
\newcommand*\ii{\mathrm{i}}

% number a single equation in unnumbered environment
\newcommand\numberthis{\addtocounter{equation}{1}\tag{\theequation}}

% shorthands
\newcommand*\ideal[1]{\left\langle #1 \right\rangle}
\newcommand*\puiseux[2]{#1\{\!\{#2\}\!\}}
\newcommand*\CCt{\puiseux{\setC}{t}}

\let\vec\mathbf
\let\idealof\trianglelefteq

% operators
\DeclareMathOperator{\Mat}{Mat}
\DeclareMathOperator{\lcm}{lcm}
\DeclareMathOperator{\Trop}{Trop}
\DeclareMathOperator{\initial}{in}
\DeclareMathOperator{\face}{face}

% use same counters for all environments
\theoremstyle{definition}
\newtheorem{definition}{Definition}
\newtheorem{theorem}[definition]{Theorem}
\newtheorem{example}[definition]{Example}
\newtheorem{remark}[definition]{Remark}

% adjust numbering
\numberwithin{definition}{section}

\title{Massively Parallel Computation of Tropical Varieties}
\subtitle{Bachelor's Thesis}
\author{Dominik Bendle}
\publishers{supervised by Janko Böhm,\\TU Kaiserslautern}
\date{\today}

\begin{document}
\pagestyle{headings}

\maketitle

\tableofcontents
\newpage

\section{Overview}

Give general introduction to topic

Describe work done at ITWM and their assistance

Mention $\mathcal G_{3,8}$ which has not been computed prior and give some properties of the
fan.

Work exposed several memory leaks and small errors in \textsc{Singular} kernel, were able
to be fixed.

Our results are implemented in computer algebra system \textsc{Singular} \cite{Singular}
with parallelization work done using GPI-Space at the ITWM.

% TODO more motivation

In the second section we will introduce the basic concepts of polyhedral fans, tropical
varieties, their various interpretations, and Puiseux series rings which we are
fundamental to our results. In the third section we describe Petri nets as a language to
formulate parallelizable workflows.

% TODO more here

\section{Computing Tropical Varieties}

Tropical geometry arises in various contexts and allows numerous interpretations, three of
which are presented here. We then focus on computing tropical varieties in polynomial
rings over fields with valuations using Newton polygons.

Computing tropical varieties follows the general structure of a fan traversal:
Interpreting the maximal cones of a polyhedral fan together with their adjacency relations
as a connected graph, we may realize the fan traversal as a simple graph traversal.

\subsection{Gröbner Fans}

One way of looking at the tropical variety of a polynomial ideal $I \idealof K[x_1, \dots,
x_n]$ is to consider its Gröbner fan. This fan partitions the space $\setR^n$ into
polyhedral cones which contain all weight vectors $w\in \setR^n$ that define the same
initial ideal of $I$. While this fan is a highly interesting object of study in and on
itself, by restricting the fan to those cones which correspond to monomial-free initial
ideals, we obtain the tropical variety of the given ideal.

We start off this section with some basic definitions from convex geometry which will lead
to the definition of the Gröbner fan of an ideal. This will then allow us to define
tropical varieties and demonstrate their connection to the Gröbner fan. For this we mainly
focus on definitions and results found in \cite{compGrobFan} and \cite{SturmGBCP}. For the
general introduction to tropical geometry later on we we also rely on
\cite{sturmMacTrop}. The first step towards defining a fan is to understand the very
important concept of polyhedra and polyhedral cones:

\begin{definition}[Polyhedral Cones]
  Recall that a set of the form $C = \{ x \in \setR^n : Ax \leq b \}$ for some matrix $A
  \in \Mat(m\times n, \setR)$ and $b \in \setR^m$ is called a polyhedron and if
  additionally $C$ is bounded it is called a polytope. If on the other hand $b=0$ then $C$
  is called a \emph{polyhedral cone}. Equivalently, a polyhedral cone $C$ is the positive
  span of finitely many vectors in $K^n$.
  \label{def:polyhedralCone}
\end{definition}

For a polyhedron $P \subseteq \setR^n$ we further define its dimension as the dimension of
the smallest affine linear space containing $P$. Consider now an element $w \in \setR^n$,
then we define the set $\face_w(P) = \{ u \in P : \ideal{w,u} \geq \ideal{w,v} \text{ for
all } v\in P\}$ where $\ideal{\cdot,\cdot}$ is the standard scalar product on $\setR^n$.
Surprisingly enough, we call a subset of $P$ of the form $\varnothing$ or $\face_w(P)$ a
\emph{face} of $P$. For example, the trivial faces of $P$ are $\varnothing$ by definition
and $P = \face_0(P)$. Clearly, a face of $P$ is a polyhedron itself, so we call it a facet
if its has dimension exactly one less than $P$. The notion of faces is important for the
next definition:

\begin{definition}[Polyhedral Fan]
  Let $\mathcal C$ be a collection of polyhedra in $\setR^n$. We call $\mathcal C$ a
  \emph{polyhedral complex} if
  \begin{enumerate}
    \item all non-empty faces of $P$ are in $\mathcal C$ for all $P \in \mathcal C$ and
    \item for any $P,Q \in \mathcal C$ the intersection $P\cap Q$ is a face of both $P$
      and $Q$.
  \end{enumerate}
  The support of $\mathcal C$ is $\bigcup_{P\in\mathcal C}P$ and we call $\mathcal C$ a
  \emph{polyhedral fan} if all polyhedra in $C$ are polyhedral cones. Further, if the
  support of a polyhedral fan is all of $\setR^n$ we call it \emph{complete}.
  \label{def:polyhedralFan}
\end{definition}

Intuitively, a polyhedral complex is a collection of polyhedra where intersecting
arbitrary elements does not produce new polyhedra and -- more importantly -- if the
intersection of two such polyhedra is a non-trivial face of both of them, they only
\enquote{touch}.

% TODO

\subsection{Tropical Prevarieties}

See \cite{compTropVar}.

\subsection{Newton Polygon Method}

We focus on a new approach introduced by Yue Ren and Tommy Hofmann which uses Newton
polygons to compute tropical varieties \cite{tropPointsLinks}. For this we fix an
algebraically closed field $K$ with non-trivial valuation $v:K\to\setR \cup \{\infty\}$
and residue field $\mathfrak R$, i.e. $\mathfrak R = R/\mathfrak m_R$ for the local ring
$R = \{ x \in K : v(x) \geq 0 \}$, with a uniformizing parameter $p\in \mathcal O_K$.
Denote $K[\vec x] = K[x_1, \ldots, x_n]$.

% TODO

\begin{definition}[Initial Forms and Tropical Varieties]
  Let $w \in \setR^n$, usually called a \emph{weight vector} and consider a polynomial $f
  \in K[\vec x]$ of the form $f = \sum_{\alpha \in \setN^n} c_\alpha \cdot \vec x^\alpha$.
  We define the \emph{initial form} of $f$ with respect to $w$ as
  \[
    \initial_w(f) = \sum_{w\alpha + \nu(c_\alpha) \text{ minimal}}
    \overline{c_\alpha \cdot p^{-\nu(c_\alpha)}} \cdot x^\alpha
    \in \mathfrak K[\vec x].
  \]
  Given an ideal $I \idealof K[\vec x]$ this naturally leads to the \emph{initial ideal}
  $\initial_w(I) = \ideal{\initial_w(f) : f\in I}$ of $I$. Finally, we may define the
  \emph{tropical variety}
  \[
    \Trop(I) := \{ w \in \setR^n \mid \initial_w(I) \text{ is monomial-free} \}.
  \]
  of $I$.
  \label{def:tropicalVarietyVal}
\end{definition}

% TODO
The definition given in \ref{def:tropicalVarietyVal} is the standard definition of
tropical varieties used in the general algorithms, however in this thesis we want to
consider a different formulation of this object. The following result is fundamental to
our approach to the computation of tropical varieties:

\begin{theorem}[Fundamental Theorem of Tropical Geometry]
  Let $K$ be an algebraically closed field with non-trivial valuation and $p\in \mathcal
  O_K$ as above. Then
  \[
    \Trop(I) = \overline{v(V(I) \cap (K^\ast)^n)},
  \]
  where $\overline{(\cdot)}$ denotes the closure in the Euclidean topology.
  % TODO
  \label{thm:fundamentalThmTropicalGeometry}
\end{theorem}

This together with our results on Newton polygons allows us to determine points in
$\Trop(I)$ without having to explicitly compute points on $V(I)$.

\subsection{Puiseux Series Rings and Puiseux Expansions}

While we did not specify the field $K$ in the previous subsection, our algorithms
explicitly deal with the field $K = \CCt$ of Puiseux series in the indeterminate
$t$. While the computations using Newton polygons only depend on the valuation of ring
elements, in some rare cases these methods can fail to produce a result, thus requiring us
to actually compute the variety of a zero-dimensional ideal. Hence we study this field and
its polynomial rings in greater detail.

First goal: slight refactoring of existing \textsc{Singular} code and removing dependency on
Magma. Use newly developed procedures for computing Puiseux expansions.

\section{Parallelization}

Use Petri nets as a framework to parallelize the traversal of tropical fans.

\subsection{Petri Nets}

Introduce Petri nets theoretically

\subsection{GPI-Space}

Implementation with extensions

\subsection{Traversing a Fan in Parallel}

We formulate the traversal algorithm into a Petri net.

Standard approach relying on a starting cone procedure and a \emph{neighbour oracle}.

\subsection{Existing Methods for Fan Traversals}

As previously mentioned we heavily rely on work done by Christian Reinbold on GIT-fans.
\dots

Introduce GIT-fans as similarly structured object

Adapting existing code base

\section{Performance and Scalability}

Benchmark speedup through parallelization.

Compute \enquote{big} examples to get a comparison.

\subsection{Tropical Grassmannians}

One important class of tropical varieties are the tropical Grassmannians which arise as
the tropical varieties of the equally -- if not more -- important Grassmannians which can
be interpreted as projective varieties. Some results on the tropical Grassmannian can be
found in \cite{tropGrass}.

\begin{definition}[Grassmannians]
  Let $n \in \setN_{>0}$ and $k \in \setN$ with $0 \leq k \leq n$, then the
  \emph{Grassmannian} of $k$-planes in $K^n$ is the set of all $k$-dimensional linear
  subspaces of $K^n$, usually denoted by $G(k, n)$.
\end{definition}

Thus, Grassmannians are in some sense a generalization of projective space and for example
we will later be able to show that in fact $G(1,n) \cong \setP^n$. Our goal now is to
give these sets the structure of a projective variety which in turn allows us to study their
tropical variety which is achieved by embedding a Grassmannian $G(k, n)$ into $\setP^{\binom
nk}-1$ via the so-called \emph{Plücker embedding}. To this end we want to regard the
corresponding linear spaces as elements of projective space, which first require some
constructions from commutative algebra.

Let $V$ and $W$ be two vector spaces and $k\in\setN$, then a multilinear map $f : V^k \to
W$ is called \emph{alternating} if for all $v_1, \dots, v_k$ with $v_i=v_j$ for some
distinct $i,j$ we have that $f(v_1, \dots, v_k) = 0$. In this setting an \emph{alternating
$k$-fold tensor product} of $V$ is a vector space $T$ together with an alternating map
$\wedge:V^k\to T, (v_1, \dots, v_k) \mapsto v_1\wedge\cdots\wedge v_k$ such that for any
other alternating map $f:W\to T$ there is a unique $\tilde f:T\to W$ such that $\tilde f
\circ \wedge = f$. It is easy to prove that such an alternating tensor product always
exists and is in fact unique up to isomorphism -- the proof is completely analogous to the
case of standard tensor products. Hence we may write $\bigwedge^k V := T$ for these vector
spaces. In the following we present some interesting and necessary properties of these
spaces which can be proved with basic knowledge of commutative algebra:

\begin{remark}
  Let $V$ be a vector space, $k \in \setN$ and $\bigwedge^k V$ the corresponding
  alternating tensor product.
  \begin{enumerate}
    \item If $v_1, \dots, v_n$ is a basis of $V$ where $n=\dim V$ as a $K$-vector space,
      then the set
      \[
        \left\{
          v_{i_1} \wedge \cdots \wedge v_{i_k} : 1\leq i_1 < \cdots < i_k \leq n
        \right\}
      \]
      will be a basis of $\bigwedge^kV$. In particular we have $\dim\bigwedge^kV = \binom
      nk$.
    \item For elements $v_1, \dots, v_k \in V$ the \emph{pure tensor} $v_1\wedge \cdots
      \wedge v_k \in \bigwedge^kV$ will be non-zero if and only if $v_1, \dots, v_k$ are
      linearly independent in $V$.
    \item It follows that for two linear independent families $v_1, \dots, v_k, w_1,
      \dots, w_k \in V$ we have that $v\wedge\cdots \wedge v_k$ and $w_1\wedge\cdots\wedge
      w_k$ are linearly dependent if and only if the families both span the same linear
      subspace of $V$.
  \end{enumerate}
\end{remark}

Consider now our setting where $V=K^n$ and let $N = \binom nk$ be the number of basis
elements in $\bigwedge^kK^n$. We may identify a $k$-dimensional linear subspace $X \in
G(k,n)$ of $V$ with a pure tensor in $\bigwedge^kV \cong K^N$ uniquely up to scalar
multiplication which -- by definition -- makes the corresponding embedding $G(k,n) \to
\setP^{N-1}$, called the \emph{Plücker embedding}, injective. Hence we can regard $G(k,n)$
as a subset of this projective space. Finally, to equip $G(k,n)$ with the structure of a
projective variety, we need this fact: For non-zero $w\in \bigwedge^kK^n$ the rank of the
$K$-linear map $f:K^n \to \bigwedge^{k+1}K^n, v \mapsto v \wedge w$ is at least $n-k$ and
equality holds if and only if $w$ is a pure tensor. Thus, using the above embedding,
$G(k,n)$ consists entirely of pure tensors $w\in G(k,n)$ which all define maps $v\mapsto
v\wedge w$ of rank at most $n-k$. This is equivalent to all $(n-k+1) \times
(n-k+1)$-minors of these maps vanishing which induces polynomial equations in the Plücker
coordinates of $\setP^{N-1}$ describing $G(k,n)$, hence it is in fact a projective
variety.

On an additional note one can cover $G(k,n)$ with copies of $K^{k(n-k)}$ to show that the
Grassmannian is an irreducible projective variety of dimension $k(n-k)$. This leads to the
following definition:

\begin{definition}[Tropical Grassmannians]
  By the above a Grassmannian can be viewed as a projective variety given by a homogeneous
  ideal
  \[
    I(G(k,n)) \idealof K[ p_{i_1, \dots, i_k} : 1\leq i_1 < \cdots < i_k \leq n ]
  \]
  with Plücker coordinates $p_{i_1,\dots,i_k}$ which defines a determinantal variety $G(k,
  n)$. We write $\mathcal G_{k,n} := \Trop(I(G(k,n)))$.
\end{definition}

\begin{figure}[htbp]
  \begin{center}
    \input{fig/scaling.tex}
  \end{center}
  \caption{Running times of computing $\mathcal{G}_{3,8}$}
  \label{fig:g38scaling}
\end{figure}

In figure~\ref{fig:g38scaling} we see that \dots and so on.

\begin{figure}[htbp]
  \begin{center}
    \input{fig/efficiency.tex}
  \end{center}
  \caption{Parallelization efficiency of computing $\mathcal{G}_{3,8}$}
  \label{fig:g38efficiency}
\end{figure}

\subsection{Practical Optimizations}

Optimization of certain edge cases

\newpage
\listoffigures
\printbibliography

\end{document}
% vim: spell spelllang=en
