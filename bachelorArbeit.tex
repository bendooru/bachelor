\documentclass[
  paper=a4,
  DIV=14,
  fontsize=12pt,
  titlepage,
  bibliography=totoc,
  pagesize=pdftex
]{scrartcl}
\usepackage[utf8]{inputenc}
\usepackage[T1]{fontenc}
\usepackage[english]{babel}
\usepackage[autostyle=true]{csquotes}
\usepackage{enumitem}
\usepackage{mathtools,amsfonts,amssymb,amsthm,mathrsfs}
\usepackage{lmodern}
\usepackage{dsfont}

\usepackage[pdftex,hidelinks]{hyperref}

\usepackage[backend=biber,style=alphabetic]{biblatex}
\addbibresource{bachelor.bib}

% serif > sans-serif
\setkomafont{sectioning}{\rmfamily\bfseries\boldmath}
\setkomafont{descriptionlabel}{\rmfamily\bfseries\boldmath}
% prefix all numbers with section
\numberwithin{figure}{section}
\numberwithin{equation}{section}
\numberwithin{table}{section}

% set default numbered list style to (i), (ii), ...
\setlist[enumerate,1]{label=(\roman*)}

% use mathds instead of mathbb
\newcommand*\setZ{\mathds{Z}}
\newcommand*\setR{\mathds{R}}
\newcommand*\setC{\mathds{C}}
\newcommand*\setQ{\mathds{Q}}
\newcommand*\setN{\mathds{N}}
\newcommand*\setK{\mathds{K}}
\newcommand*\setP{\mathds{P}}
\newcommand*\ii{\mathrm{i}}

% number a single equation in unnumbered environment
\newcommand\numberthis{\addtocounter{equation}{1}\tag{\theequation}}

% shorthands
\newcommand*\ideal[1]{\left\langle #1 \right\rangle}
\newcommand*\puiseux[2]{#1\{\!\{#2\}\!\}}
\newcommand*\CCt{\puiseux{\setC}{t}}

\let\vec\mathbf
\let\idealof\trianglelefteq

% operators
\DeclareMathOperator{\Mat}{Mat}
\DeclareMathOperator{\lcm}{lcm}
\DeclareMathOperator{\Trop}{Trop}
\DeclareMathOperator{\initial}{in}

% use same counters for all environments
\theoremstyle{definition}
\newtheorem{definition}{Definition}
\newtheorem{theorem}[definition]{Theorem}
\newtheorem{example}[definition]{Example}

% adjust numbering
\numberwithin{definition}{section}

\title{Massively Parallel Computation of Tropical Varieties}
\subtitle{Bachelor's Thesis}
\author{Dominik Bendle}
\publishers{supervised by Janko Böhm,\\TU Kaiserslautern}
\date{\today}

\begin{document}
\pagestyle{headings}

\maketitle

\tableofcontents
\newpage

\section{Overview}

Give general introduction to topic

Describe work done at ITWM and their assistance

Mention $\mathcal G_{3,8}$ which has not been computed prior and give some properties of the
fan.

Work exposed several memory leaks and small errors in \textsc{Singular} kernel, were able
to be fixed.

Our results are implemented in computer algebra system \textsc{Singular} \cite{Singular}
with parallelization work done using GPI-Space at the ITWM.

% TODO more motivation

In the second section we will introduce the basic concepts of polyhedral fans, tropical
varieties, their various interpretations, and Puiseux series rings which we are
fundamental to our results. In the third section we describe Petri nets as a language to
formulate parallelizable workflows.

% TODO more here

\section{Computing Tropical Varieties}

Tropical geometry arises in various contexts and allows numerous interpretations, three of
which are presented here. We then focus on computing tropical varieties in polynomial
rings over fields with valuations using Newton polygons.

Computing tropical varieties follows the general structure of a fan traversal:
Interpreting the maximal cones of a polyhedral fan together with their adjacency relations
as a connected graph, we may realize the fan traversal as a simple graph traversal.

\subsection{Gröbner Fans}

We start off the section with some basic definitions from convex geometry. See for example
\cite{compGrobFan} and \cite{SturmGBCP}.

\begin{definition}[Polyhedral Cones and Fans]
  Fix a field $K$ and consider the vector space $K^n$. Recall that a set of the form $C =
  \{ x \in K^n : Ax \leq b \}$ for some matrix $A \in \Mat(m\times n, K)$ and $b \in K^m$
  is called a polyhedron or polytope if $C$ is bounded. If on the other hand $b=0$ then
  $C$ is called a \emph{polyhedral cone}. Equivalently, a polyhedral cone $C$ is the
  positive span of finitely many vectors in $K^n$.
  \label{def:polyhedralFan}
\end{definition}
Now \ldots

\begin{definition}[Initial Forms and Tropical Varieties]
  Let $w \in \setR^n$, usually called a \emph{weight vector} and consider a polynomial $f
  \in K[\vec x]$ of the form $f = \sum_{\alpha \in \setN^n} c_\alpha \cdot \vec x^\alpha$.
  We define the \emph{initial form} of $f$ with respect to $w$ as
  \[
    \initial_w(f) = \sum_{w\alpha + \nu(c_\alpha) \text{ minimal}}
    \overline{c_\alpha \cdot p^{-\nu(\alpha)}} \cdot x^\alpha
    \in \mathfrak K[\vec x].
  \]
  Given an ideal $I \idealof K[\vec x]$ this naturally leads to the \emph{initial ideal}
  $\initial_w(I) = \ideal{\initial_w(f) : f\in I}$ of $I$. Finally, we may define the
  \emph{tropical variety}
  \[
    \Trop(I) := \{ w\in \setR^n \mid \initial_w(I) \text{ is monomial-free} \}.
  \]
  of $I$.
  \label{def:tropicalVariety}
\end{definition}

\subsection{Tropical Prevarieties}

\subsection{Newton Polygon Method}

We focus on a new approach introduced by Yue Ren and Tommy Hofmann which uses Newton
polygons to compute tropical varieties \cite{tropPointsLinks}. For this we fix an
algebraically closed field $K$ with non-trivial valuation $v:K\to\setR \cup \{\infty\}$
and residue field $\mathfrak R$, i.e. $\mathfrak R = R/\mathfrak m_R$ for the local ring
$R = \{ x \in K : v(x) \geq 0 \}$, with a uniformizing parameter $p\in \mathcal O_K$.
Denote $K[\vec x] = K[x_1, \ldots, x_n]$.

% TODO
The definition given in \ref{def:tropicalVariety} is the standard definition of tropical
varieties used in the general algorithms, however in this thesis we want to consider a
different formulation of this object. The following result is fundamental to our approach
to the computation of tropical varieties:

\begin{theorem}[Fundamental Theorem of Tropical Geometry]
  Let $K$ be an algebraically closed field with non-trivial valuation and $p\in \mathcal
  O_K$ as above. Then
  \[
    \Trop(I) = \overline{v(V(I) \cap (K^\ast)^n)},
  \]
  where $\overline{(\cdot)}$ denotes the closure in the Euclidean topology.
  % TODO
  \label{thm:fundamentalThmTropicalGeometry}
\end{theorem}

This together with our results on Newton polygons allows us to determine points in
$\Trop(I)$ without having to explicitly compute points on $V(I)$. See \cite{compTropVar}.

\subsection{Puiseux Series Rings and Puiseux Expansions}

While we did not specify the field $K$ in the previous subsection, our algorithms
explicitly deal with the field $K = \CCt$ of Puiseux series in the indeterminate
$t$. While the computations using Newton polygons only depend on the valuation of ring
elements, in some rare cases these methods can fail to produce a result, thus requiring us
to actually compute the variety of a zero-dimensional ideal. Hence we study this field and
its polynomial rings in greater detail.

First goal: slight refactoring of existing \textsc{Singular} code and removing dependency on
Magma. Use newly developed procedures for computing Puiseux expansions.

\section{Parallelization}

Use Petri nets as a framework to parallelize the traversal of tropical fans.

\subsection{Petri Nets}

Basically: GPI-Space and Petri nets.

\subsection{Traversing a Fan in Parallel}

We formulate the traversal algorithm into a Petri net.

Standard approach relying on a starting cone procedure and a \emph{neighbour oracle}.

\subsection{Existing Methods for Fan Traversals}

As previously mentioned we heavily rely on work done by Christian Reinbold on GIT-fans.
\dots

Introduce GIT-fans as similarly structured object

Adapting existing code base

\section{Performance and Scalability}

Benchmark speedup through parallelization.

Compute \enquote{big} examples to get a comparison.

In particular consider important examples of the tropical Grassmanians

\subsection{Tropical Grassmannians}

\begin{definition}[Grassmannians]
  Let $n \in \setN_{>0}$ and $k \in \setN$ with $0 \leq k \leq n$, then the
  \emph{Grassmannian} of $k$-planes in $K^n$ is the set of all $k$-dimensional linear
  subspaces of $K^n$, usually denoted by $G(k, n)$.
\end{definition}

Our goal is to give these sets the structure of an affine variety which in turn allows us
to study their tropical variety. This is achieved by embedding a Grassmannian $G(k, n)$
into $K^{\binom nk}$ via the so-called \emph{Plücker embedding}:

\dots
% TODO

\begin{definition}[Tropical Grassmannians]
  By the above a Grassmannian can be viewed as a homogeneous ideal $G(k,n) \idealof K[ p_U
  : U \subset \{0, \dots, n \}, |U|=k ]$ with Plücker coordinates $p_U$ which defines a
  determinantal variety $V(G(k, n))$. We write $\mathcal G_{k,n} := \Trop(G(k,n))$.
\end{definition}

\begin{figure}[htbp]
  \begin{center}
    \input{fig/scaling.tex}
  \end{center}
  \caption{Running times of computing $\mathcal{G}_{3,8}$}
  \label{fig:g38scaling}
\end{figure}

In figure~\ref{fig:g38scaling} we see that \dots and so on.

\begin{figure}[htbp]
  \begin{center}
    \input{fig/efficiency.tex}
  \end{center}
  \caption{Parallelization efficiency of computing $\mathcal{G}_{3,8}$}
  \label{fig:g38efficiency}
\end{figure}

\subsection{Practical Optimizations}

Optimization of certain edge cases

\newpage
\listoffigures
\printbibliography

\end{document}
% vim: spell spelllang=en
