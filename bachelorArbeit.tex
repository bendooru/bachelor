\documentclass[
  paper=a4,
  titlepage,
  bibliography=totoc,
  listof=totoc,
  pagesize=pdftex
]{scrartcl}
\usepackage[utf8]{inputenc}
\usepackage[T1]{fontenc}
\usepackage[english]{babel}
\usepackage[autostyle=true]{csquotes}
\usepackage{enumitem}
\usepackage{mathtools,amsfonts,amssymb,amsthm,mathrsfs}
\usepackage{lmodern}
\usepackage{dsfont}

\usepackage{algpseudocode}
\algrenewcommand\algorithmicrequire{\textbf{Input:}}
\algrenewcommand\algorithmicensure{\textbf{Output:}}

\usepackage[
  activate={true,nocompatibility},
  kerning=true,
  spacing=true
]{microtype}

\usepackage{pgf,tikz}
\usepackage[pdftex,hidelinks]{hyperref}

\usepackage[
  backend=biber,
  style=alphabetic,
  maxnames=5,
  maxalphanames=5,
  isbn=false
]{biblatex}
\addbibresource{bachelor.bib}

% serif > sans-serif
\setkomafont{sectioning}{\rmfamily\bfseries\boldmath}
\setkomafont{descriptionlabel}{\rmfamily\bfseries\boldmath}
% prefix all numbers with section
\numberwithin{figure}{section}
\numberwithin{equation}{section}
\numberwithin{table}{section}

% set default numbered list style to (i), (ii), ...
\setlist[enumerate,1]{label=(\roman*)}

% use mathds instead of mathbb
\newcommand*\setZ{\mathds{Z}}
\newcommand*\setR{\mathds{R}}
\newcommand*\setC{\mathds{C}}
\newcommand*\setQ{\mathds{Q}}
\newcommand*\setN{\mathds{N}}
\newcommand*\setK{\mathds{K}}
\newcommand*\setA{\mathds{A}}
\newcommand*\setP{\mathds{P}}
\newcommand*\setT{\mathds{T}}
\newcommand*\ii{\mathrm{i}}

% number a single equation in unnumbered environment
\newcommand\numberthis{\addtocounter{equation}{1}\tag{\theequation}}
\newcommand*\dotcup{\mathbin{\dot{\cup}}}

% shorthands
\newcommand*\ideal[1]{\left\langle #1 \right\rangle}
\newcommand*\puiseux[2]{#1\{\!\{#2\}\!\}}
\newcommand*\CCt{\puiseux{\setC}{t}}

\let\vec\mathbf
\let\idealof\trianglelefteq

% operators
\DeclareMathOperator{\Mat}{Mat}
\DeclareMathOperator{\lcm}{lcm}
\DeclareMathOperator{\Trop}{Trop}
\DeclareMathOperator{\trop}{trop}
\DeclareMathOperator{\initial}{in}
\DeclareMathOperator{\face}{face}
\DeclareMathOperator{\GR}{GR}
\DeclareMathOperator{\conv}{conv}

% use same counters for all environments
\theoremstyle{definition}
\newtheorem{definition}{Definition}
\newtheorem{theorem}[definition]{Theorem}
\newtheorem{proposition}[definition]{Proposition}
\newtheorem{example}[definition]{Example}
\newtheorem{remark}[definition]{Remark}
\newtheorem{lemma}[definition]{Lemma}
\newtheorem{algo}[definition]{Algorithm}

% adjust numbering
\numberwithin{definition}{section}

\title{Massively Parallel Computation of Tropical Varieties}
\subtitle{Bachelor's Thesis}
\author{Dominik Bendle}
\publishers{supervised by Janko Böhm,\\TU Kaiserslautern}
\date{\today}

\begin{document}
\pagestyle{headings}

\maketitle

\tableofcontents
\newpage

\section{Overview}

Give general introduction to topic

Describe work done at ITWM and their assistance

Mention $\mathcal G_{3,8}$ which has not been computed prior and give some properties of the
fan.

Work exposed several memory leaks and small errors in \textsc{Singular} kernel, were able
to be fixed.

Our results are implemented in computer algebra system \textsc{Singular} \cite{Singular}
with parallelization work done using GPI-Space at the ITWM.

% TODO more motivation

In the second section we will introduce the basic concepts of tropical geometry and
tropical varieties. We will discuss several different approaches of computing tropical
varieties, including considering the Gröbner fan of an ideal and the new Newton polygon
method described in \cite{tropPointsLinks}. Consequently we will take a look at Puiseux
series rings which are special fields with valuation and fundamental to our results. In
the third section we describe Petri nets as a language to formulate parallelizable
workflows.

% TODO more here

\section{Computing Tropical Varieties}

Tropical geometry is a relatively new field in mathematics that arises from numerous
contexts and connects various branches of mathematics in exciting and novel ways. We shall
provide a basic introduction to this topic while focusing on the computational aspects
which will lead to main interest of this thesis -- the recently developed techniques using
Newton polygons and valuations.

Over the course of this section we need some basic definitions from algebraic geometry: We
fix a field $K$ which is usually assumed to be algebraically closed: The \emph{affine
space} over $K$ of dimension $n$ is
\[
  \setA_K^n = \setA^n = \left\{ (a_1,\dots,a_n) : a_i \in K\right\} = K^n
\]
and the \emph{projective space} over $K$ of dimension $n$ is $\setP^n_K = \setP^n =
(K^{n+1}\setminus\{0\})/\sim$ with equivalence relation $v \sim \lambda v$ for all $v \in
K^n\setminus \{0\}$ and $\lambda \in K^*$. Elements $a \in \setP^n$ are usually written as
homogeneous coordinates $(a_1:\cdots:a_n)$ for a representative $(a_0, \dots, a_n)$. An
\emph{affine variety} is the common zero locus of all the polynomials of a polynomial
ideal $I \idealof K[x_1, \dots, x_n]$, that means sets of the form
\[
  X := V(I) = \{ a \in \setA^n : f(a) = 0 \text{ for all $f\in I$} \} \subset \setA^n.
\]
In the special case of a homogeneous ideal $I \idealof K[x_0,\dots,x_n]$, which means $I$
is generated by homogeneous polynomials -- we may define \emph{projective varieties}
\[
  X := V(I) = \left\{ (a_0:\cdots:a_n) \in \setP^n : f(a_0, \dots, a_n) = 0
    \text{ for all $f \in I$}
  \right\} \subset \setP^n.
\]
The homogeneity assumption is necessary since otherwise $f(a)$ being zero is not
independent from the representative $a$ of $a\in \setP^n$.

\subsection{Basic Tropical Geometry}
\label{sec:tropIntro}

The standard approach to tropical geometry is to consider the semiring $(\setR \cup
\{\infty\}, \min, +)$ with the classical addition replaced with taking the minimum and
classical multiplication replaced with addition. An in-depth introduction to tropical
geometry and tropical varieties is given by \cite{sturmMacTrop}, similar notions focused
on computation can be found in \cite{compTropVar}.

We denote by $\oplus$ and $\odot$ the \enquote{addition} and \enquote{multiplication}
defined above, so -- just to reiterate -- we have that
\[
  x \oplus y = \min(x,y)
  \qquad \text{and} \qquad
  x \odot y = x+y
\]
for elements $x,y\in \setR\cup\{\infty\}$, for example $4\oplus9 = 4$ and $4\odot9 = 13$.
It is easy to verify that the usual arithmetic laws are still valid: $\oplus$ and $\odot$
are commutative and satisfy distributivity. Further, there is a unique neutral  element
for both operations: For all $x\in \setR\cup \{\infty\}$ it is true that
\[
  x \oplus \infty = x
  \qquad \text{and} \qquad
  x \odot 0 = x.
\]
As in proper rings, the neutral element of $\oplus$ has the annihilating property with
respect to $\odot$, i.e.\ $x\odot \infty = \infty$. An important property of this set is
that, while elements in $\setR$ admit a multiplicative $\odot$-inverse (the traditional
additive inverse), there is in general no additive $\oplus$-inverse. For example the
equation $4\oplus x = 5$ clearly has no solution. In this setting all ring axioms except
the existence of an additive inverse are satisfied: Sets like this are in fact called
semirings, hence we call $(\setR\cup \{\infty\}, \oplus, \odot)$ the \emph{tropical
semiring} which is sometimes also referred to a as the \emph{min-plus algebra}. By
replacing taking minima with maxima, we obtain the isomorphic max-plus algebra which is
also used as the underlying structure in tropical geometry by some authors. These special
operations of addition and multiplication naturally lead to the notion of polynomials over
the tropical semiring.

\begin{definition}[Tropical Polynomials]
  Let $x_1, \dots, x_k$ be variables with values in $\setR\cup\{\infty\}$, then due to
  commutativity we can write arbitrary products of variables as
  \[
    x_{i_1} \odot x_{i_2} \odot \cdots \odot x_{i_m}
    = x_1^{\alpha_1} \cdots x_k^{\alpha_k}
  \]
  for suitable $\alpha_1, \dots, \alpha_k \in \setN$. Unsurprisingly we call this a
  \emph{tropical monomial}. Translating back into the traditional operations shows that
  the tropical monomials
  \[
    x_1^{\alpha_1} \cdots x_k^{\alpha_k} =
    \underbrace{x_1\odot\cdots\odot x_1}_{\alpha_1 \text{ times}}
    \odot\cdots\odot
    \underbrace{x_k\odot\cdots\odot x_k}_{\alpha_k \text{ times}}
    = \alpha_1x_1 + \cdots + \alpha_kx_k
  \]
  are just linear polynomials with integer coefficients. A \emph{tropical polynomial} is
  then just a finite $\setR$-linear combination of tropical monomials:
  \[
    p = a_1 \odot x_1^{\alpha_{1,1}}\cdots x_k^{\alpha_{1,k}} \oplus \cdots \oplus
    a_m \odot x_1^{\alpha_{m,1}}\cdots x_k^{\alpha_{m,k}}
  \]
  with exponents $\alpha_{i,j} \in \setN$, $1\leq i \leq m, 1\leq j \leq k$ and real
  coefficients $a_i \in \setR$. Note that if $a_i = 0$ for some $i$ we have that $a_i\odot
  x_1^{\alpha_{i,1}}\cdots x_k^{\alpha_{i,k}} = x_1^{\alpha_{i,1}}\cdots
  x_k^{\alpha_{i,k}}$ so we can omit $0$-coefficients. As usual, we call a monomial in $p$
  together with its coefficient a \emph{term}.
  \label{def:tropPoly}
\end{definition}

\begin{figure}[tbh]
  \centering
  \begin{tikzpicture}
    \draw[->] (-0.1,0) -- (5.5,0) node[right] {$x$};
    \draw[->] (0,-0.1) -- (0,5.5) node[left] {$p(x)$};
    \begin{scriptsize}
      \foreach \x in {1, 2, 3, 4, 5}
      {
        \draw (0.1,\x) -- (-0.1,\x) node[left] {$\x$};
        \draw (\x,0.1) -- (\x,-0.1) node[below] {$\x$};
      }
    \end{scriptsize}

    \draw[dashed] (-0.05,-0.1) -- (2.75,5.5);
    \draw[dashed] (-0.1,1.9) -- (3.5,5.5);
    \draw[dashed] (-0.1,5) -- (5.5,5);
    \draw[thick] (-0.05,-0.1) -- (2,4) -- (3,5) -- (5.5,5);
  \end{tikzpicture}
  \caption{Graph of the tropical polynomial $p=x^2\oplus 2\odot x \oplus 5$}
  \label{fig:tropPolyPlot}
\end{figure}

By again translating back to the traditional operations, evaluating this polynomial at
elements in $\setR^k$ yields a function
\[
  p : \setR^k \to \setR, (x_1, \dots, x_k) \mapsto
  \min\left\{
    a_i + \sum_{j=1}^k \alpha_{i,j}x_j : 1 \leq i \leq m
  \right\}
\]
which is continuous, piecewise linear with finitely many pieces and concave, meaning that
for any $a,b \in \setR^k$ the equality $p(\frac12(a+b)) \geq \frac12(p(a)+p(a))$ holds.
Consider for example the polynomial $p = x^2 \oplus 2\odot x \oplus 5$ in the single
variable $x$. As we see in figure~\ref{fig:tropPolyPlot}, each monomial of $p$ corresponds
to a linear piece of the graph and the function $p$ is in fact concave.

With the newly established notion of tropical polynomials the next step towards defining
tropical varieties is to consider special sets $X \subset \setR^k$ induced by a tropical
polynomial. Whereas the main object of study in classical algebraic geometry are the roots
or vanishing loci of polynomials and -- by extension -- polynomial ideals, being zero is
not a particularly interesting property of a tropical polynomial. Similarly, evaluating to
the additive identity is equally uninteresting: If $p$ is not already the constant
$\infty$-function, then $p(x)=\infty$ if and only if $x=\infty$. Instead, we use the fact
that the induced function is piecewise linear:

\begin{definition}[Tropical Hypersurface]
  Let $p = \bigoplus_{i=1}^k a_i \odot x_1^{\alpha_{i,1}}\cdots x_k^{\alpha_{i,k}}$ be a
  tropical polynomial in the variables $x_1, \dots, x_k$ with induced piecewise linear
  function $p:\setR^k \to \setR$. For an element $a\in \setR^k$ the value $p(a)$ is the
  minimum over the terms of $p$; we set
  \[
    V(p) = \left\{
      a \in \setR^k :
      \text{ the minimum $p(a)$ is attained by at least two terms of $p$}
    \right\}
  \]
  and call it the \emph{tropical hypersurface} of $p$. Equivalently, a point $a\in
  \setR^k$ lies in $V(p)$ if and only if $p$ is not linear at $a$. With this
  interpretation in mind, the set $V(p)$ is also called the \emph{corner locus} of $p$.
  \label{def:tropHypersurface}
\end{definition}

Again looking at the tropical polynomial $p$ plotted in figure~\ref{fig:tropPolyPlot} we
see that $p$ is not linear at precisely the points $V(p) = \{ 2, 3 \}$. An interesting
property is that by this definition the tropical hypersurface given by a monomial contains
no points whatsoever. This will become important later on.

\begin{figure}[tbh]
  \centering
  \begin{tikzpicture}[thick]
    \draw (0,0) -- (2,2);
    \draw (2,2) -- (2,4.5);
    \draw (2,2) -- (4.5,2);
  \end{tikzpicture}
  and quadric
  % TODO
  \caption{A generic tropical line and quadric in $\setR^2$}
  \label{fig:tropLineQuad}
\end{figure}

Since monomials and powers of variables are well-defined, we can assign a \emph{degree} to
a tropical polynomial in the usual sense, hence we may carry over the naming conventions
for hypersurfaces in standard algebraic geometry: A hypersurface given by a degree 1
polynomial is called a tropical hyperplane, for degrees 2, 3 and so on they are called
tropical quadrics and cubics respectively. In figure~\ref{fig:tropLineQuad} we see the
hypersurfaces given by \dots % TODO
As a final note, since tropical division is well-defined for non-infinite elements we may
also consider tropical polynomials in negative exponents later on.

\subsection{Valuations and Puiseux Series}

Most of the following definitions and algorithms will make use of fields equipped with a
non-trivial valuation: We fix a field $K$ and recall that a \emph{valuation} is a map
$\nu:K\to \setR \cup \{\infty\}$ that satisfies the following properties: For any $a, b
\in K$ we have
\begin{enumerate}
  \item $\nu(a) = \infty$ if and only if $a=0$,
  \item $\nu(ab) = \nu(a)+\nu(b)$ and
  \item $\nu(a+b) \geq \min\{\nu(a), \nu(b)\}$ with equality if $a\neq b$.
\end{enumerate}
A valuation is called \emph{non-trivial} if it is not the constant 0-function on $K^* =
K\setminus\{0\}$. Such a field with valuation defines a local ring $R_\nu = \{ x \in K :
\nu(x) \geq 0 \}$ with unique maximal ideal $\mathfrak m_{R_\nu} = \{ x \in K : \nu(x) > 0
\}$. The \emph{residue field} of $K$ is then defined to be the residue field $\mathfrak K
= R_\nu/\mathfrak m_{R_\nu}$ of $R_\nu$.
% TODO: uniformizing parameter
While there are various fields admitting a non-trivial valuation that
are suitable for tropical computations, the most important one for our case is the field
of Puiseux series:

\begin{definition}
  Let $K'$ be a field. The field of \emph{Puiseux series} $K = \puiseux{K'}{t}$ with
  coefficients in $K'$ in the indeterminate $t$ is the set of all formal power series
  \[
    f = c_1 t^{a_1} + c_2 t^{a_2} + c_3 t^{a_3} + \cdots
  \]
  with $c_k \in K'$ for all $k \in \setN$ and rational numbers $a_1 < a_2 < a_3 < \cdots$
  that have a common denominator. Hence, the series can be rewritten as
  \[
    f = \sum_{k = k_0}^\infty c'_k t^{\frac kN}
  \]
  with suitable $c'_k \in K'$ for all $k\geq k_0$ and the common denominator $N \in
  \setN$.
  \label{def:puiseux}
\end{definition}

By considering the rings of formal Laurent series $K'((t^{\frac1n}))$ for $n \in \setN$ in
the indeterminate $t^{\frac1n}$, we can see that
\[
  K = \puiseux{K'}t = \bigcup_{n \in \setN} K'((t^{\frac1n})).
\]
The most important use case will involve $K'$ being algebraically closed, in particular we
usually choose $K'=\setC$.

\subsection{Gröbner Fans}
\label{sec:grobFan}

One way of looking at the tropical variety of a polynomial ideal $I \idealof K[\vec x] :=
K[x_1, \dots, x_n]$ is to first consider its Gröbner fan. This fan partitions the space
$\setR^n$ into polyhedral cones which contain all weight vectors $w\in \setR^n$ that
define the same initial ideal of $I$. While this fan is a highly interesting object of
study in and on itself, by restricting the fan to those cones which correspond to
monomial-free initial ideals, we obtain the tropical variety of the given ideal.

We start off this section with some basic definitions from convex geometry which will lead
to the definition of the Gröbner fan of an ideal. This will then allow us to define
tropical varieties and demonstrate their connection to the Gröbner fan. For this we mainly
focus on definitions and results found in \cite{compGrobFan} and \cite{SturmGBCP}. The
first step towards defining a fan is to understand the very important concept of polyhedra
and polyhedral cones:

\begin{definition}[Polyhedral Cones]
  Recall that a set of the form $C = \{ x \in \setR^n : Ax \leq b \}$ for some matrix $A
  \in \Mat(m\times n, \setR)$ and $b \in \setR^m$ is called a polyhedron and if
  additionally $C$ is bounded it is called a polytope. If on the other hand $b=0$ then $C$
  is called a \emph{polyhedral cone}. Equivalently, a polyhedral cone $C$ is the positive
  span of finitely many vectors in $\setR^n$.
  \label{def:polyhedralCone}
\end{definition}

For a polyhedron $P \subseteq \setR^n$ we further define its dimension as the dimension of
the smallest affine linear space containing $P$. Consider now an element $w \in \setR^n$,
then we define the set $\face_w(P) = \{ u \in P : \ideal{w,u} \geq \ideal{w,v} \text{ for
all } v\in P\}$ where $\ideal{\cdot,\cdot}$ is the standard scalar product on $\setR^n$.
Surprisingly enough, we call a subset of $P$ of the form $\varnothing$ or $\face_w(P)$ a
\emph{face} of $P$. For example, the trivial faces of $P$ are $\varnothing$ by definition
and $P = \face_0(P)$. Clearly, a face of $P$ is a polyhedron itself, so we call it a facet
if it has dimension exactly one less than $P$. The notion of faces is important for the
next definition:

\begin{definition}[Polyhedral Fan]
  Let $\mathcal C$ be a collection of polyhedra in $\setR^n$. We call $\mathcal C$ a
  \emph{polyhedral complex} if
  \begin{enumerate}
    \item all non-empty faces of $P$ are in $\mathcal C$ for all $P \in \mathcal C$ and
    \item for any $P,Q \in \mathcal C$ the intersection $P\cap Q$ is a face of both $P$
      and $Q$.
  \end{enumerate}
  The support of $\mathcal C$ is $\bigcup_{P\in\mathcal C}P$ and we call $\mathcal C$ a
  \emph{polyhedral fan} if all polyhedra in $C$ are polyhedral cones. Furthermore, if the
  support of a polyhedral fan is all of $\setR^n$ we call it \emph{complete}.
  \label{def:polyhedralFan}
\end{definition}

Intuitively, a polyhedral complex is a collection of polyhedra where intersecting
arbitrary elements does not produce new polyhedra and -- more importantly -- if the
intersection of two such polyhedra is a non-trivial face of both of them, they only
\enquote{touch}.

Now, given a polynomial ring $K[\vec x]$ a total \emph{term} or \emph{monomial ordering}
$\prec$ on the monomials $K[\vec x]$ is an ordering such that $1 \prec \vec x^\alpha =
x_1^{\alpha_1}\cdots x_n^{\alpha_n}$ for all $\alpha \in \setN^n$ and if $\vec x^\alpha
\prec \vec x^\beta$ for some $\alpha, \beta \in \setN^n$ then also $\vec x^{\alpha+\gamma}
\prec \vec x^{\beta+\gamma}$ for all $\gamma \in \setN^n$. Hence we can order the non-zero
terms of any non-zero polynomial $f \in K[\vec x]$ which defines a unique maximal
\emph{initial term} $\initial_\prec(f)$. Correspondingly, for an ideal $I \idealof K[\vec
x]$ we define the \emph{initial ideal} of $I$ as $\initial_\prec(I) = \ideal{\initial_\prec(f)
: f \in I}$. Initial ideals are used to define one of the most important concepts in
computer algebra: A finite subset $\mathcal G \subset I$ is called a \emph{Gröbner basis}
of the ideal $I$ with respect to $\prec$ if its initial terms $\mathcal G' = \{
\initial_\prec(g) : g \in \mathcal G \}$ generate $\initial_\prec(I)$. Further, if the
elements of $\mathcal G'$ are irredundant $\mathcal G$ is called \emph{minimal} and if for
any $g, g' \in \mathcal G$ no terms of $g$ are divisible by $\initial_\prec(g')$ then
$\mathcal G$ is \emph{reduced}. While there are infinitely many possible term orderings in
polynomials rings with more than one variable, one can show that for a fixed $I \idealof
K[\vec x]$ there are only finitely many different initial ideals using the Noetherian
property of polynomial rings over fields.

Using this argument we are interested in grouping all term orderings into finitely many
classes, but this first requires a way to identify a given ordering with a more tangible
description. The idea is to represent term orderings by weight vectors:

\begin{definition}[Initial Forms]
  Given a $\vec w \in \setR^n$ we define the \emph{initial form} $\initial_{\vec w}(f)$ of
  $f = \sum_{\alpha\in\setN^n} c_\alpha \vec x^\alpha \in K[\vec x]$ with respect to $\vec
  w$ to be the sum of all non-zero terms $c_\alpha \vec x^\alpha$ of $f$ where $\vec
  w\cdot \alpha$ is maximal. Analogously to initial terms this defines an initial ideal
  $\initial_{\vec w}(I)$ for $I \idealof K[\vec x]$.
  \label{def:initFormG}
\end{definition}

Note that initial ideals of this form need not be monomial ideals -- in fact the weight
vectors for which this is not the case will be of interest later on. If necessary, this
can be fixed by introducing a \enquote{tie breaker} ordering: Given any total term
ordering $\prec$ and any weight vector $\vec w\in\setR^n$ with non-negative components we
may define a new term ordering $\prec_{\vec w}$ by
\[
  \vec x^\alpha \prec_{\vec w} \vec x^\beta
  \quad\iff\quad
  \vec w \cdot \alpha < \vec w\cdot \beta
  \text{ or }
  \left(
    \vec w \cdot \alpha = \vec w\cdot \beta
    \text{ and }
    \vec x^\alpha \prec \vec x^\beta
  \right)
\]
where the non-negativity constraint is required to ensure that $\prec_{\vec w}$ is also a
total ordering, meaning that 1 will be smaller than any other monomial with respect to
$\prec_{\vec w}$. Among some other nice properties, the most important result obtained
from these constructions is the following:

\begin{proposition}[{\cite[Proposition~1.11]{SturmGBCP}}]
  For any term ordering $\prec$ and any ideal $I \idealof K[\vec x]$ there is a weight
  vector $\vec w \in \setR^n_{\geq0}$ with $\initial_\prec(I) = \initial_{\vec w}(I)$.
\end{proposition}

\begin{figure}[tbh]
  \centering
  some Gröbner region
  % TODO
  \caption{Gröbner region of $\ideal{-}$}
  \label{fig:gr}
\end{figure}

Hence it makes sense to only consider initial forms and ideals given by non-negative
weight vectors $\vec w \in \setR^n_{\geq0}$. Extending this, for a given $I \idealof
K[\vec x]$ we call
\[
  \GR(I) = \left\{
    \vec w \in \setR^n : \exists \vec w' \in \setR^n_{\geq0} :
    \initial_{\vec w}(I) = \initial_{\vec w'}(I)
  \right\}
\]
the \emph{Gröbner region} of $I$. From the definition it is clear that $\setR^n_{\geq0}
\subset \GR(I)$ for any ideal. I general, $\GR(I)$ will be a proper subset of $\setR^n$ as
we see in figure \ref{fig:gr} for the ideal $I=\ideal{-}$, but under certain conditions we
get a nice Gröbner region:

\begin{proposition}[{\cite[Proposition~1.12]{SturmGBCP}}]
  If $I \idealof K[\vec x]$ is a homogeneous ideal with respect to some positive grading,
  then $\GR(I) = \setR^n$.
\end{proposition}

This warrants us to pay special attention to homogeneous ideals, in fact in this thesis we
will limit ourselves to computing tropical varieties of such homogeneous ideals.

% TODO

\subsection{Tropical Prevarieties}
\label{sec:tropPre}

After this light introduction to tropical arithmetic and some convex geometry we now shift
our focus to more general polynomials and ideals. The idea is to combine the fan structure
of tropical varieties we derived in section~\ref{sec:grobFan} with our knowledge on
tropical hypersurfaces from section~\ref{sec:tropIntro}. Here we will introduce the
general procedure of computing a tropical variety by computing cones in the corresponding
fan and finding its neighbor cones.

We fix a field $K$ which admits a usually non-trivial valuation $\nu : K \to \setR \cup
\{\infty\}$. The usual setting is to consider the \emph{Laurent polynomial} ring $K[\vec
x^\pm] = K[x_1^\pm, \dots, x_n^\pm]$ -- the ring of polynomials in $x_1, \dots, x_n$ with
integral but not necessarily non-negative exponents. Ideals and polynomials in this ring
define zero loci much in the same manner as for standard polynomial rings, however as
monomials are now invertible, evaluation at zero is no longer well-defined. As such, we
consider the \emph{very affine variety} of an ideal $I \idealof K[\vec x^\pm]$ as a subset
of the \emph{algebraic torus} $\setT^n := {(K^*)}^n = {(K \setminus \{0\})}^n$:
\[
  X := V(I) = \left\{ a \in \setT^n : f(a) = 0 \text{ for all $f \in I$} \right\}.
\]
In this setting -- although not limited to Laurent polynomials -- we now want to study
initial ideals. To define them in a useful way, we first need to establish a link between
classical and tropical polynomials: this is where the valuation $\nu:K\to\setR$ will be
used.

% TODO: initial forms coincide only in max-plus algebra, so change?
\begin{definition}[Tropicalization and Initial Forms]
  Let $f \in K[\vec x^\pm]$ be a Laurent polynomial and $\nu : K \to \setR$ a valuation as
  in the above setting. Write $f = \sum_{\alpha \in \setZ^n} c_\alpha \vec x^\alpha$. The
  \emph{tropicalization} $\trop(f)$ of $f$ is then the tropical polynomial
  \[
    \trop(f) = \bigoplus_{\alpha\in\setZ^\alpha} \nu(c_\alpha)
    \odot x_1^{\alpha_1}\odot\cdots \odot x_n^{\alpha_n}
  \]
  which then defines a function $\trop(f) : \setR^n \to \setR, \vec w \mapsto
  \trop(f)(\vec w)$. For a weight vector $\vec w \in \setR^n$ we now define the
  \emph{initial form} of $f$ with respect to $\vec w$ as
  \[
    \initial_{\vec w}(f) =
    \sum_{ \substack{
        \alpha \in \setZ^n \\
        \nu(c_\alpha) + \vec w\cdot \alpha = \trop(f)(\vec w)
    }} \overline {c_\alpha \cdot t^{-\nu(c_\alpha)} } \vec x^\alpha
    \in \mathfrak K[\vec x^\pm].
  \]
  In other words, $\initial_{\vec w}(f)$ consists of all the terms of $f$ where
  $\nu(c_\alpha)+\vec w\cdot \alpha$ is minimal.
  \label{def:initialId}
\end{definition}

We note here that the notations from Definitions~\ref{def:initFormG} and
\ref{def:initialId} clash. While we chose the minimum-convention for tropical arithmetic
to stay consistent with the important literature, both notions of initial forms coincide
in the max-plus algebra and with the trivial valuation $\nu(x) = 0$ for $x\neq-\infty$.
Since section~\ref{sec:grobFan} is only given for context and the important definitions in
convex geometry, $\initial(f)$ will always refer to the initial forms defined here.

\dots % TODO

\begin{definition}[Tropical Varieties]
  Let $I \idealof K[\vec x]$ be an ideal, then the \emph{tropical variety} of $I$ is
  defined as the set
  \[
    \Trop(I) = \bigcap_{f \in I} \Trop(f)
  \]
  % TODO
  \label{def:tropicalVariety}
\end{definition}

\dots % TODO

\begin{theorem}[Fundamental Theorem of Tropical Geometry,
  {\cite[Theorem~3.2.5]{sturmMacTrop}}]
  Let $K$ be an algebraically closed field with non-trivial valuation, $I \idealof K[\vec
  x^\pm]$ and $X = V(I) \subset \setT^n$ its very affine variety in the affine torus, then
  the following three subsets of $\setR^n$ coincide:
  \begin{enumerate}
    \item the tropical variety $\Trop(I)$ as defined in
      definition~\ref{def:tropicalVariety},
    \item the closure of the set of all $\vec w \in \setR^n$ with $\initial_{\vec w}(I)
      \neq \ideal1$, i.e.\ the initial ideals which do not contain monomials and
    \item the set of coordinate-wise valuations of points in $X$
      \[
        \nu(X) = \{ (\nu(x_1), \dots, \nu(x_n)) : (x_1, \dots, x_n) \in X \}.
      \]
  \end{enumerate}
  Here, closure is the Euclidean closure in $\setR^n$.
  \label{thm:fundamentalThm}
\end{theorem}

% much TODO

\subsection{Newton Polygon Method}

We focus on a new approach introduced by Yue Ren and Tommy Hofmann which uses Newton
polygons to compute zero-dimensional tropical varieties \cite{tropPointsLinks}. While the
general traversal procedures are given in section~\ref{sec:tropPre}, we now want to study
different methods for computing non-trivial points in the tropical variety and links
between cones in it, which allows us to generate a starting and neighbor cones more
efficiently. For this we again fix an algebraically closed field $K$ with non-trivial
valuation $v:K\to\setR \cup \{\infty\}$ and residue field $\mathfrak K$ with uniformizing
parameter $p$.

While the previously discussed methods dealt with tropical varieties as finite
intersections of tropical prevarieties or the set of all weight vectors that define
monomial-free initial ideals, the focus of this section -- and in fact of the entire
thesis -- is the interpretations as the coordinate-wise image $\Trop(I) = \nu(V(I))$ as we
have seen in theorem~\ref{thm:fundamentalThm}. The main idea of Ren and Hofmann's results
is to apply this interpretation to zero-dimensional ideals and their triangular
decompositions: A set of polynomials $\{ f_1, \dots, f_n \} \subseteq K[x_1, \dots, x_n]$
is called a \emph{triangular set} if
\[
  f_i \in K[x_1, \dots, x_i] / K[x_1, \dots, x_{i-1}]
\]
for each $i \in \{1, \dots, n\}$. Zero loci of ideals generated by triangular sets can be
computed very easily as only univariate methods and substitution are required. It is thus
important to be able to reduce general zero-dimensional ideals to triangular sets:

\begin{proposition}
  Let $I \idealof K[\vec x]$ be a zero-dimensional ideal, then there exist triangular sets
  $F_1, \dots, F_s \subset K[\vec x]$ such that
  \begin{enumerate}
    \item $\sqrt I = \bigcap_{i=1}^s \sqrt{\ideal{F_i}}$ and
    \item $\ideal{F_i} + \ideal{F_j} = \ideal1$ for any $i\neq j$.
  \end{enumerate}
  The collection $F_1, \dots, F_s$ is called a \emph{triangular decomposition} of $I$.
  \label{prp:triang}
\end{proposition}

In particular $V(I) = V(F_1) \dotcup \cdots \dotcup V(F_s)$ for a triangular decomposition
$F_1, \dots, F_s$ which allows us to easily determine roots of a zero-dimensional ideal
$I$. Moreover, we are actually only interested in the valuations of the roots of
univariate polynomials over $K$. Luckily, there is a way to determine these without
needing to explicitly compute any roots. For this, we study the Newton polygon of a
polynomial:

\begin{definition}[Newton Polyon]
  Let $f \in K[x_k]$ be a univariate polynomial of the form $f = \sum_{i=0}^d c_i x_k^i$
  for $c_0, \dots, c_d \in K$. The \emph{Newton polygon} or \emph{extended Newton
  polyhedron} is defined as
  \[
    \Delta(f) := \conv \left(
      \left\{ (i, \nu(c_i)) : i = 0, \dots, d \right\}
      \cup \{ (0, \infty) \}
    \right)
    \subseteq \setR^2
  \]
  with $\conv(\cdot)$ being the convex hull of the given set. Similarly, for a
  multivariate polynomial $f \in K[x_1, \dots, x_k]$ written as $f = \sum_{i=0}^d f_i
  \cdot x_k^i$ for some $f_0, \dots, f_d \in K[x_1, \dots, x_{k-1}]$ and a weight $\vec w
  \in \setR^{k-1}$, the \emph{expected Newton polygon} of $f$ at $\vec w$ is defined to be
  \[
    \Delta_{\vec w}(f) := \conv\left(
      \left\{ (i, \trop(f_i)(\vec w)) : i = 0, \dots, d \right\}
      \cup \{ (0, \infty) \}
    \right).
  \]
  Finally, $f$ has a \emph{unique Newton polygon} at $\vec w$ if the initial form
  $\initial_{\vec w}(f)$ is a monomial for all vertices $(i, \trop(f_i)(\vec w)) \in
  \Delta_{\vec w}(f)$. We denote by $\Lambda(f)$ and $\Lambda_{\vec w}(f)$ the sets of
  negatives of all slopes of $\Delta(f)$ and $\Delta_{\vec w}(f)$ respectively.
  \label{def:newtonPoly}
\end{definition}

The notion of unique Newton polygons helps us to establish a link between valuations of
roots and polynomials evaluated at these roots:

\begin{proposition}
  For a polynomial $f \in K[x_1, \dots, x_k]$ and a weight $\vec w \in \setR^{k-1}$ the
  following are equivalent:
  \begin{enumerate}
    \item $f$ has a unique Newton polygon at $\vec w$
    \item for all $z \in K^{k-1}$ with valuation $\nu(z) = (\nu(z_1), \dots, \nu(z_{k-1}))
      = (w_1, \dots, w_{k-1}) = \vec w$ we have $\Delta(f(z, x_k)) = \Delta_{\vec w}(f)$.
  \end{enumerate}
  \label{prp:expectedNewt}
\end{proposition}

In other words if $f$ has a unique Newton polygon we may regard its expected Newton
polygon instead of evaluating $f$ at a partial root and vice versa. Now, the use of Newton
polygons becomes apparent with the following lemma which draws the connection between
valuations of roots and the slopes of the Newton polygons:

\begin{lemma}
  Let $f \in K[x]$ and $e$ be an edge of the Newton polygon $\Delta(f)$ connecting
  vertices $(r,x)$ and $(s, y)$ with slope $-m = \frac{y-x}{s-r}$. Then $f$ has exactly
  $s-r$ roots with valuation $m$.
  In other words if $f$ has a unique Newton polygon we may regard its expected Newton
  polygon instead of evaluating $f$ at a partial root and vice versa.
  \label{lem:newtPolyRoots}
\end{lemma}

Proposition~\ref{prp:expectedNewt} and Lemma~\ref{lem:newtPolyRoots} naturally lead to the
formulation of the following algorithm to compute tropical varieties of zero-dimensional
ideals given by a triangular set. By Proposition~\ref{prp:triang} this then enables us to
compute the tropical variety of any zero-dimensional ideal:

\begin{algo}[Zero-dimensional tropical varieties] $ $
  \begin{algorithmic}[1]
    \Require a triangular set $F = \{f_1, \dots, f_n\} \subseteq K[x_1, \dots, x_n]$ with
    zero-dimensional affine variety $V(F) \subseteq \setT^n$.
    \Ensure $\vec w \in \Trop(F)$.
    \State Pick $w_1 \in \Lambda(f_1)$.
    \label{alg:pick1}
    \For{$i = 2, \dots, n$}
      \If{$f_i$ has unique Newton polygon at $(w_1, \dots, w_{i-1})$}
        \State Pick $w_i \in \Lambda_{(w_1, \dots, w_{i-1})}(f_i)$.
        \label{alg:pick2}
      \Else
        \State Compute root $(c_1, \dots, c_{i-1}) \in V(f_1, \dots, f_{i-1})$
        \State Pick $w_i \in \Lambda(f_i(c_1, \dots, c_{i-1}, x_i))$.
        \label{alg:pick3}
      \EndIf
    \EndFor
    \State \textbf{return} $\vec w = (w_1, \dots, w_n)$;
  \end{algorithmic}
\end{algo}

If, instead of picking a single slope in lines \ref{alg:pick1}, \ref{alg:pick2} and
\ref{alg:pick3}, we iterate over the entire respective sets of slopes the algorithm will
compute the entire tropical variety $\Trop(F)$.

\section{Parallelization}

Use Petri nets as a framework to parallelize the traversal of tropical fans.

\subsection{Petri Nets}

Introduce Petri nets theoretically

\subsection{GPI-Space}

Implementation with extensions

\subsection{Traversing a Fan in Parallel}

Computing tropical varieties in our setting follows the general structure of a fan
traversal: Interpreting the maximal cones of a polyhedral fan together with their
adjacency relations as a connected graph, we may realize the fan traversal as a simple
graph traversal.

We formulate the traversal algorithm formulated in section~\ref{sec:tropPre} into a Petri
net.

Standard approach relying on a starting cone procedure and a \emph{neighbour oracle}.

\subsection{Existing Methods for Fan Traversals}

As previously mentioned we heavily rely on work done by Christian Reinbold on GIT-fans.
\dots

Introduce GIT-fans as similarly structured object

Adapting existing code base

\section{Performance and Scalability}

Benchmark speedup through parallelization.

Compute \enquote{big} examples to get a comparison.

\subsection{Tropical Grassmannians}

One important class of tropical varieties are the tropical Grassmannians which arise as
the tropical varieties of the equally -- if not more -- important Grassmannians which can
be interpreted as projective varieties. Some results on the tropical Grassmannian can be
found in \cite{tropGrass}.

\begin{definition}[Grassmannians]
  Let $n \in \setN_{>0}$ and $k \in \setN$ with $0 \leq k \leq n$, then the
  \emph{Grassmannian} of $k$-planes in $K^n$ is the set of all $k$-dimensional linear
  subspaces of $K^n$, usually denoted by $G(k, n)$.
\end{definition}

Thus, Grassmannians are in some sense a generalization of projective space and for example
we will later be able to show that in fact $G(1,n) \cong \setP^n$. Our goal now is to
give these sets the structure of a projective variety which in turn allows us to study their
tropical variety which is achieved by embedding a Grassmannian $G(k, n)$ into $\setP^{\binom
nk}-1$ via the so-called \emph{Plücker embedding}. To this end we want to regard the
corresponding linear spaces as elements of projective space, which first require some
constructions from commutative algebra.

Let $V$ and $W$ be two vector spaces and $k\in\setN$, then a multilinear map $f : V^k \to
W$ is called \emph{alternating} if for all $v_1, \dots, v_k$ with $v_i=v_j$ for some
distinct $i,j$ we have that $f(v_1, \dots, v_k) = 0$. In this setting an \emph{alternating
$k$-fold tensor product} of $V$ is a vector space $T$ together with an alternating map
$\wedge:V^k\to T, (v_1, \dots, v_k) \mapsto v_1\wedge\cdots\wedge v_k$ such that for any
other alternating map $f:W\to T$ there is a unique $\tilde f:T\to W$ such that $\tilde f
\circ \wedge = f$. It is easy to prove that such an alternating tensor product always
exists and is in fact unique up to isomorphism -- the proof is completely analogous to the
case of standard tensor products. Hence we may write $\bigwedge^k V := T$ for these vector
spaces. In the following we present some interesting and necessary properties of these
spaces which can be proved with basic knowledge of commutative algebra:

\begin{remark}
  Let $V$ be a vector space, $k \in \setN$ and $\bigwedge^k V$ the corresponding
  alternating tensor product.
  \begin{enumerate}
    \item If $v_1, \dots, v_n$ is a basis of $V$ where $n=\dim V$ as a $K$-vector space,
      then the set
      \[
        \left\{
          v_{i_1} \wedge \cdots \wedge v_{i_k} : 1\leq i_1 < \cdots < i_k \leq n
        \right\}
      \]
      will be a basis of $\bigwedge^kV$. In particular we have $\dim\bigwedge^kV = \binom
      nk$.
    \item For elements $v_1, \dots, v_k \in V$ the \emph{pure tensor} $v_1\wedge \cdots
      \wedge v_k \in \bigwedge^kV$ will be non-zero if and only if $v_1, \dots, v_k$ are
      linearly independent in $V$.
    \item It follows that for two linear independent families $v_1, \dots, v_k, w_1,
      \dots, w_k \in V$ we have that $v\wedge\cdots \wedge v_k$ and $w_1\wedge\cdots\wedge
      w_k$ are linearly dependent if and only if the families both span the same linear
      subspace of $V$.
  \end{enumerate}
\end{remark}

Consider now our setting where $V=K^n$ and let $N = \binom nk$ be the number of basis
elements in $\bigwedge^kK^n$. We may identify a $k$-dimensional linear subspace $X \in
G(k,n)$ of $V$ with a pure tensor in $\bigwedge^kV \cong K^N$ uniquely up to scalar
multiplication which -- by definition -- makes the corresponding embedding $G(k,n) \to
\setP^{N-1}$, called the \emph{Plücker embedding}, injective. Hence we can regard $G(k,n)$
as a subset of this projective space. Finally, to equip $G(k,n)$ with the structure of a
projective variety, we need this fact: For non-zero $w\in \bigwedge^kK^n$ the rank of the
$K$-linear map $f:K^n \to \bigwedge^{k+1}K^n, v \mapsto v \wedge w$ is at least $n-k$ and
equality holds if and only if $w$ is a pure tensor. Thus, using the above embedding,
$G(k,n)$ consists entirely of pure tensors $w\in G(k,n)$ which all define maps $v\mapsto
v\wedge w$ of rank at most $n-k$. This is equivalent to all $(n-k+1) \times
(n-k+1)$-minors of these maps vanishing which induces polynomial equations in the Plücker
coordinates of $\setP^{N-1}$ describing $G(k,n)$, hence it is in fact a projective
variety.

On an additional note one can cover $G(k,n)$ with copies of $K^{k(n-k)}$ to show that the
Grassmannian is an irreducible projective variety of dimension $k(n-k)$. This leads to the
following definition:

\begin{definition}[Tropical Grassmannians]
  By the above a Grassmannian can be viewed as a projective variety given by a homogeneous
  ideal
  \[
    I(G(k,n)) \idealof K[ p_{i_1, \dots, i_k} : 1\leq i_1 < \cdots < i_k \leq n ]
  \]
  with Plücker coordinates $p_{i_1,\dots,i_k}$ which defines a determinantal variety $G(k,
  n)$. We write $\mathcal G_{k,n} := \Trop(I(G(k,n)))$.
\end{definition}

\subsection{Runtime Measurements}

\begin{figure}[htbp]
  \begin{center}
    \input{fig/scaling.tex}
  \end{center}
  \caption{Running times of computing $\mathcal{G}_{3,8}$}
  \label{fig:g38scaling}
\end{figure}

In figure~\ref{fig:g38scaling} we see that \dots and so on.

\begin{figure}[htbp]
  \begin{center}
    \input{fig/efficiency.tex}
  \end{center}
  \caption{Parallelization efficiency of computing $\mathcal{G}_{3,8}$}
  \label{fig:g38efficiency}
\end{figure}

\subsection{Practical Optimizations}

Optimization of certain edge cases

Mention improvements to \textsc{Singular} kernel here?

\newpage
\listoffigures
\newpage% only maybe
\printbibliography

\end{document}
% vim: spell spelllang=en
